\documentclass[aspectratio=43, notes]{beamer}
\usepackage{hyperref}
\usepackage{csquotes,lipsum}
\usepackage{array}
\usepackage{multirow}
\renewcommand{\mkblockquote}[4]{#1#2\par\hfill#4#3}
%\mode<presentation>
\usetheme{DSTG}
\usepackage{graphics}

\usepackage{xcolor, colortbl}
\usepackage{booktabs}
\usepackage{tikz}
\usetikzlibrary{arrows.meta, positioning, matrix, fit, chains, calc, shapes.geometric, shapes.misc, arrows, decorations.markings}


\usepackage{ifthen}
\usepackage{pgf}
\usepackage{pgfpages}
\usepackage[utf8]{inputenc}
\usepackage{natbib}
\usepackage{enumitem}
\usepackage{animate}

\usepackage{standalone}
%\usepackage{algorithmicx}
%\usepackage{neuralnetwork}
%\usepackage[table]{xcolor}
%\usepackage{pgfplots, siunitx}
\usepackage[most]{tcolorbox}

\usepackage{pgfplots}
% (use this `compat` level or higher to make use of the "advanced" axis
%  label positioning)
\pgfplotsset{compat=1.3}

% for beamer [handout] option only
%\pgfpagesuselayout{2 on 1}[a4paper,border shrink=5mm]

% animate code
%\newcount\myangle

\setbeamertemplate{caption}[numbered]
	

%% For confusion matrices commands
\newcommand\MyBox[2]{
	\fbox{\lower0.75cm
		\vbox to 1.7cm{\vfil
			\hbox to 1.7cm{\hfil\parbox{1.4cm}{#1\\#2}\hfil}
			\vfil}%
	}%
}
%% End confusion matrix commands

%\newcommand*\revealcline{\noalign{\vskip\arrayrulewidth}}
%\newcommand*\nextrow[1]
%{\\\cline{#1}\noalign{\vskip1ex}\cline{#1}\revealcline}
%\newcount\ccellA
%\newcommand*\ccell[2]
%{%
%	\def\tmpa{}%
%	\ccellA=1
%	\loop
%	\edef\tmpa{\unexpanded\expandafter{\tmpa\cellcolor{ccellcolor}}}%
%	\fi
%	\ifnum#2>\ccellA
%	\edef\tmpa{\unexpanded\expandafter{\tmpa&}}%
%	\repeat
%	\tmpa
%}


% Define the default classification marking to go on frames that do not have an explicit classification
\classification{\U}

% The title slide contents are defined here, so we use ``maketitle'' as with a conventional slide. 
% If titleclassification is not explicitly defined, then the ''maximum'' classification found is placed on the title slide.
\titleclassification{\O}
\author{Simon Hosking and Melissa Stolar}

	\title{Team performance metrics}
\subtitle{Communication dynamics as a predictor of team state}
%\logo{}
\institute{Human Factors\\
	Aerospace Division}

\begin{document}


% For presentations without explicit release conditions just use make title
\maketitle

% but for presentation with release - edit the following in an appropriate manner - and comment out \maketitle!
%{
%\usebackgroundtemplate{\includegraphics[width=\paperwidth,height=\paperheight]{DST_FrontPageBackground}}
%\frame{
	%\centering
		%\LARGE{\textcolor{DSTOpresentationtitle}{\inserttitle}} \\[12pt]
		%\normalsize{\textcolor{DSTOauthor}{\insertauthor}} \par
		%\normalsize{\textcolor{DSTOdate}{\insertdate}} \par
%
%\begingroup
%\setbeamercolor{block title}{bg=DSTOreleaseboxheaderBG,fg=DSTOreleaseboxheaderFG}%bg=background, fg= foreground
%\setbeamercolor{block body}{bg=DSTOreleaseboxheaderBG,fg=DSTOreleaseboxheaderFG}%bg=background, fg= foreground
%\begin{block}{\small Release Statement:}\tiny%
	%Release statement goes here\\
	%More release text goes here if required
%\end{block}
%\endgroup	
%}
%}
%\begin{frame}
%\begin{figure}
%	\centering
%	\includegraphics[width=0.7\linewidth]{figures/systems}
%	%\caption{}
%	\label{fig:systems}
%\end{figure}
%\end{frame}

\begin{frame}
\begin{figure}
	\includestandalone[scale = .9]{tikzFigs/system}%     without .tex extension
	% or use \input{mytikz}
	\caption{tizk version of systems}
	\label{fig:tsystems}
\end{figure}
\end{frame}


%\begin{frame}
%\begin{figure}
%	\centering
%	\includegraphics[width=0.7\linewidth]{figures/metrics}
%	%\caption{}
%	\label{fig:metrics}
%\end{figure}
%\end{frame}

\begin{frame}
\begin{figure}
	\includestandalone[scale = .5]{tikzFigs/perfMetrics}%     without .tex extension
	% or use \input{mytikz}
	\caption{tizk version of metrics}
	\label{fig:tmetrics}
\end{figure}
\end{frame}



\begin{frame}
\frametitle{Introduction}
\begin{figure}[hb]
	\begin{center}
		\includegraphics[width=4cm]{figures/abm1.png}\hspace{.2cm}\includegraphics[width=4cm]{figures/debrief.png}\\
		\vspace{.2cm}
		\includegraphics[width=3cm]{figures/aim.png}
	\end{center}
	\label{fig:setup} 
\end{figure}
\end{frame}


\begin{frame}
\frametitle{Scenario and objectives}
\begin{itemize}[label=$\bullet$]
	\item Tactical command and control team
	\item 24 waves of 6-8 aircraft; wave initiated every 60s. 
	\item Each wave contains one HOSTILE aircraft only. 
	\item Team responsible for:
	\begin{itemize}[label=$\bullet$]
		\item Monitoring all traffic through the airspace.
	    \item Communicating mission-related information with other team members via push-to-talk microphones. 
		\item Identifying and engaging HOSTILE aircraft using provided ROE.
	\end{itemize}
\end{itemize}
\end{frame}


\begin{frame}
\frametitle{Surveillance Operator (SO)}
\small
\begin{columns}
\column{0.5\textwidth}
\begin{itemize}[label=$\bullet$]
	\item Could see aircraft in entire airspace
	\item Had access to point of origin and route information
	\item Responsible for: 
	\begin{itemize} [label=$\bullet$]
		\item Identifying aircraft not on approved routes
		\item Declaring SUSPECT aircraft
		\item Reporting SUSPECT aircraft to TD
		\item Maintaining awareness  of SUSPECT aircraft
	\end{itemize}
\end{itemize}
\column{0.5\textwidth}
\begin{figure}[h]
	\begin{center}
		\includegraphics[trim=1cm 2.4cm 6cm .3cm,clip,width=5.0cm]{figures/screenShots/SO_HW20s.eps}
	\end{center}
\end{figure}
\end{columns}
\end{frame}


\begin{frame}
\frametitle{Tactical Director (TD)}
\small
\begin{columns}
\column{0.5\textwidth}
\begin{itemize}[label=$\bullet$]
	\item Could see aircraft in controlled airspace
	\item Had access to speed and altitude information
	\item Responsible for:
	\begin{itemize}[label=$\bullet$]
		\item Interrogating aircraft to determine speed and altitude
		\item Prioritising threat level of SUSPECT aircraft
		\item Directing FC to intercept aircraft on basis of priorities
	\end{itemize} 
\end{itemize}
\column{0.5\textwidth}
\begin{figure}[h]
	\begin{center}
		\includegraphics[trim=1cm 2.4cm 6cm .3cm,clip,width=5.0cm]{figures/screenShots/TD_HW_inside45s.eps}\\
	\end{center}
\end{figure}
\end{columns}
\end{frame}


\begin{frame}
\frametitle{Fighter Controller (FC)}
\small
\begin{columns}
\column{0.5\textwidth}
\begin{itemize}[label=$\bullet$]
	\item Could see aircraft in smaller radius
	\item Required directions from TD to locate and intercept SUSPECT aircraft
	\item Had access to IFF and manoeuvre information
	\item Responsible for:
	\begin{itemize}[label=$\bullet$]
		\item Moving fighters to intercept aircraft according to TDs priorities
		\item Interrogating aircraft to determine IFF and Manoeuvre
		\item Engaging hostile aircraft. 
	\end{itemize}
\end{itemize}
\column{0.5\textwidth}
\begin{figure}[h]
	\begin{center}
		\includegraphics[trim=1cm 2.4cm 6cm .3cm,clip,width=5.0cm]{figures/screenShots/FC_HW_inview.eps}\\
	\end{center}
\end{figure}
\end{columns}
\end{frame}


\begin{frame}
\frametitle{Rules of Engagement}
\begin{itemize}	
	\item SUSPECT DECLARATION
	\begin{itemize}[label=$\bullet$]
		\item Inside controlled airspace, and 
		\item Inbound to the base on an unapproved route. 
	\end{itemize}
	\item HOSTILE DECLARATION	
	\begin{itemize}[label=$\bullet$]
		\item SUSPECT declaration, plus \ldots
		\item Speed $>50$
		\item Altitude $>50$
		\item Friendly IFF $= 0$ ("lack of friendly")
		\item Manoeuvre $= 1$ ("evasive")
	\end{itemize}
\end{itemize}
\end{frame}

\begin{frame}
\begin{figure}
	\includestandalone[width=\textwidth]{tikzFigs/decisionTree}%     without .tex extension
	% or use \input{mytikz}
	\caption{tizk version of decision tree}
	\label{fig:tsystems}
\end{figure}
\end{frame}

\begin{frame}
\frametitle{Method}
\begin{columns}
	\column{0.5\textwidth}
	\begin{itemize}[label=$\bullet$]
		\item 10 teams of 3 participants randomly allocated to TD, SO, or FC. 
		\item 3 sessions per week for 4 weeks
		\item One level of workload per session with one repetition
	\end{itemize}
	\column{0.5\textwidth}
	\begin{itemize}[label=$\bullet$]
		\item 3 workload conditions:
		\begin{itemize}[label = $\bullet$]
			\item Low (6 aircraft)
			\item Medium (7 aircraft)
			\item High (8 aircraft)
		\end{itemize}

	\end{itemize}
\end{columns}
\end{frame}

%\begin{frame}
%\begin{figure}[h]
%	\begin{center}
%		\includegraphics[trim=1cm 2.4cm 6cm .3cm,clip,width=3.0cm]{figures/screenShots/SO_HW20s.eps}\hspace{.2cm}
%		\includegraphics[trim=1cm 2.4cm 6cm .3cm,clip,width=3.0cm]{figures/screenShots/TD_HW_inside45s.eps}\\
%		\vspace{.2cm}
%		\hspace{.07cm}\includegraphics[trim=1cm 2.4cm 6cm .3cm,clip,width=3.0cm]{figures/screenShots/FC_HW45.eps}\hspace{.2cm}
%		\includegraphics[trim=1cm 2.4cm 6cm .3cm,clip,width=3.0cm]{figures/screenShots/FC_HW_inview.eps}
%		\vspace{.3cm}
%		\caption{}
%	\end{center}
%\end{figure}
%
%\end{frame}

%\begin{frame}
%\begin{figure}[hb]
%\begin{center}
%	%	\framebox[0.60\textwidth]{%
%	%		\input{setup.tex}
%	%	}
%	\includegraphics[scale = .5]{setup.eps}
%\end{center}
%\label{fig:setup} 
%\end{figure}
%\tiny{Hosking, S.G., Best, C.J., Jia, D., Ross, P., \& Watkinson, P. (2018). Validating team communication data using a transmission-duration threshold and voice activity detection algorithm. \emph{Behavior Research Methods, 51}, 384-397.}
%\end{frame}

%\begin{frame}
%\begin{figure}
%	\includestandalone[scale = .5]{tikzFigs/setup}%     without .tex extension
%	% or use \input{mytikz}
%	\caption{tizk setup}
%	\label{fig:tsetup}
%\end{figure}
%\end{frame}

%\begin{frame}
%\begin{figure}
%	\includestandalone[scale = .5]{tikzFigs/setup}%     without .tex extension
%	% or use \input{mytikz}
%	\caption{tizk setup}
%	\label{fig:tsetup}
%\end{figure}
%\end{frame}

\begin{frame}
\begin{columns}
	\column{0.4\textwidth}
\begin{figure}
	\includegraphics[scale = .5]{tikzFigs/images/setup.png}%     without .tex extension
	% or use \input{mytikz}
	%\caption{tizk setup}
	\label{fig:tsetup}
\end{figure}
\column{0.6\textwidth}
\begin{itemize}[label=$\bullet$]
	\item DIS-to-audio 
	\item Discretise utterances
	\item Noise filter - transmission duration threshold and VAD (Hosking et al., 2018)
	\item Data transformed to ordered and nominal communication sequences
\end{itemize}
\end{columns}
\vspace{2cm}
\footnotesize{Hosking, S.G., Best, C.J., Jia, D., Ross, P., \& Watkinson, P. (2018). Validating team communication data using a transmission-duration threshold and voice activity detection algorithm. \emph{Behavior Research Methods, 51}, 384-397.}
\end{frame}

\begin{frame}
\frametitle{}
\begin{figure}
	\centering
	\includegraphics[width=0.6\linewidth]{figures/fourState.pdf}
	\label{fig:4state}
\end{figure}
\end{frame}

\begin{frame}
\frametitle{RQA outputs} 
\begin{itemize}[label=$\bullet$]
	\item \%REC: the percentage of recurrent points out of all possible points in the RP. 
	\item \%DET: the predictability, or the patterned structure of recurrence of the system under study (\emph{determinism}) -- differentiates time series that may appear complex or irregular yet possess predictable structure from those that are truly random or stochastic processes
	\item MAXLINE: the longest repeated string of data points in the time series -- indicative of \emph{stability} because more stable system dynamics are less likely to be interrupted by a perturbation.
	\item ENTROPY: Shannon information entropy of a histogram of diagonal line lengths --  a measure of the complexity of the time series.
\end{itemize}
\end{frame}

%\begin{frame}
%\begin{figure}
%	\vspace{-4cm}
%	\includestandalone[scale = .5]{tikzFigs/comp}%     without .tex extension
%	% or use \input{mytikz}
%	\caption{tizk comp}
%	\label{fig:tsetup}
%\end{figure}
%\end{frame}

%\begin{frame}
%\animategraphics[width=\textwidth]{12}{figures/gif/frame-}{0}{36}
%\end{frame}

%% Basic rectangles
%\begin{frame}
%\begin{figure}
%	\includestandalone[width=6cm]{tikzFigs/basicRectangles}%     without .tex extension
%	% or use \input{mytikz}
%	\caption{Basic rectangles}
%	\label{fig:basicRectanges}
%\end{figure}
%\end{frame}


%neural network
%\begin{frame}
%\begin{figure}
%	\includestandalone[width=6cm]{tikzFigs/neuralNetwork}%     without .tex extension
%	% or use \input{mytikz}
%	\caption{Neural network}
%	\label{fig:neuralNetwork}
%\end{figure}
%\end{frame}


%% n x n confusion matrix
%\begin{frame}
%\begin{figure}
%	\includestandalone[width=6cm]{tikzFigs/nXnConfusionMatrix}%     without .tex extension
%	% or use \input{mytikz}
%	\caption{n * n confusion matrix}
%	\label{fig:nXnConfusionMatrix}
%\end{figure}
%\end{frame}




\begin{frame}
\frametitle{}
\begin{figure}
	\centering
	\includegraphics[width=0.6\linewidth]{figures/RP.eps}
	\label{fig:RP}
\end{figure}
\end{frame}

%% computers
%\begin{frame}
%\begin{figure}
%	\includestandalone[width=6cm]{tikzFigs/computers}%     without .tex extension
%	% or use \input{mytikz}
%	\caption{Computers}
%	\label{fig:computers}
%\end{figure}
%\end{frame}



\begin{frame}
\frametitle{}
\begin{figure}
	\centering
	\includegraphics[width=0.6\linewidth]{figures/detXweekXhitClass.eps}
	\label{fig:dweekHitclass}
\end{figure}
\end{frame}


\begin{frame}
\frametitle{}
\begin{figure}
\centering
\includegraphics[width=0.6\linewidth]{figures/detXworkloadXhitClass.eps}
\label{fig:dworkloadHitclass}
\end{figure}
\end{frame}

\begin{frame}
\begin{figure}
	\includestandalone[scale = .5]{tikzFigs/window}%     without .tex extension
	% or use \input{mytikz}
	%\caption{tizk version of metrics}
	\label{fig:tmetrics}
\end{figure}
\end{frame}


%
%\begin{frame}
%%\begin{figure}
%%	\only<1>{\includestandalone[scale=.5]{tikzFigs/window}}%
%%	\only<2>{\includestandalone[scale=.5]{tikzFigs/window}}%
%%	\only<3>{\includestandalone[scale=.5]{tikzFigs/window}}%
%%\end{figure}
%
%\tikzset{
%	invisible/.style={opacity=0},
%	visible on/.style={alt={#1{}{invisible}}},
%	alt/.code args={<#1>#2#3}{%
%		\alt<#1>{\pgfkeysalso{#2}}{\pgfkeysalso{#3}} % \pgfkeysalso doesn't change the path
%	},
%}
%
%\begin{tikzpicture}
%%    \node[anchor=south west,inner sep=0] at (0,0) {\includegraphics[width=\textwidth]{/home/simon/Documents/work/training/learningExperiment/manuscript/Presentation/tikzFigs/images/fourState.pdf}};
%%    \draw[red,ultra thick,rounded corners] (7.5,5.3) rectangle (9.4,6.2);
%
%\node[anchor=south west,inner sep=0] (image) at (0,0) {\includegraphics[width=0.5\textwidth]{/home/simon/Documents/work/training/learningExperiment/manuscript/Presentation/tikzFigs/images/fourState.pdf}};
%\begin{scope}[x={(image.south east)},y={(image.north west)}]
%\draw[red,ultra thick,rounded corners, visible on=<1->] (0.175,0.2) rectangle (0.4,0.97);
%\draw[red,ultra thick,rounded corners, visible on=<2->] (0.275,0.2) rectangle (0.5,0.97);
%\draw[red,ultra thick,rounded corners, visible on=<3->] (0.375,0.2) rectangle (0.6,0.97);
%%\only<4->\draw[red,ultra thick,rounded corners] (0.475,0.2) rectangle (0.7,0.97);
%%\only<5->\draw[red,ultra thick,rounded corners] (0.575,0.2) rectangle (0.8,0.97);
%%\only<6->\draw[red,ultra thick,rounded corners] (0.675,0.2) rectangle (0.9,0.97);
%% \draw[help lines,xstep=.1,ystep=.1] (0,0) grid (1,1);
%%\foreach \x in {0,1,...,9} { \node [anchor=north] at (\x/10,0) {0.\x}; }
%%\foreach \y in {0,1,...,9} { \node [anchor=east] at (0,\y/10) {0.\y}; }
%\end{scope}
%\end{tikzpicture}
%\end{frame}

% AMIMATION
%\begin{frame}[t,fragile]{Protein}
%\animatevalue<2-22>{\myangle}{0}{20}
%\begin{figure}
%	\includestandalone[scale = .5]{tikzFigs/animate}%     without .tex extension
%	% or use \input{mytikz}
%	\caption{tizk version of metrics}
%	\label{fig:tmetrics}
%\end{figure}
%\end{frame}

%\def\r{3}
%\def\s{0}
%
%\begin{frame}<1-10>
%\pgfmathsetmacro{\r}{2 + \r}
%\pgfmathsetmacro{\s}{.1 + \s}
%\global\let\r=\r
%\global\let\s=\s
%\begin{tikzpicture}[every node/.style={fill=green!50!white}]
%
%\onslide<1->\node (b) at (-1,0) [ rectangle]  {R};
%\onslide<2->\node (r) at  (1.4,0) [circle, minimum size=\r pt,xshift=-5mm] {S};
%\onslide<3->\node (d) at (2.5,-2) [circle,xshift=-5mm] {PRD};
%\onslide<3->\draw [->] (b.east) -- ($(b.east)!\s!(d.west)$);
%\onslide<4->\node (c) at (2.5,2) [circle, xshift=-5mm] {PSD};
%\onslide<4->\draw [->] (b.east) -- ($(b.east)!\s!(c.west)$);
%
%\end{tikzpicture}
%\end{frame}

% K-fold
\begin{frame}
\begin{itemize}[label=$\bullet$]
	\item Shuffle the dataset randomly.
	\item Split the dataset into k subsets
	\begin{itemize}[label=$\bullet$]
		\item Use one of the k-subsets as test data set
		\item Use the remainder of the k-subsets as training data set
		\item Fit a model on the training set and evaluate it on the test set
		\item Evaluate model performance score
	\end{itemize}
    \item Repeat k times
	\item Determine average model performance using the sample of model evaluation scores ($\sum E_k/k$)
\end{itemize}
\end{frame}

%\begin{frame}
%\begin{figure}
%\includestandalone[width=6cm]{tikzFigs/kFold}%     without .tex extension
%% or use \input{mytikz}
%\caption{k - Fold}
%\label{fig:kFold}
%\end{figure}
%\end{frame}


\begin{frame}
\begin{figure}
\includestandalone[width=9cm]{tikzFigs/3-folds}%     without .tex extension
% or use \input{mytikz}
\caption{k-fold cross-validation}
\label{fig:kfolds}
\end{figure}
\end{frame}


%\begin{frame}
%\definecolor{ccellcolor}{HTML}{d09000}
%\begin{tabular}[c]{l *5{|p{2em}}|}
%	& \multicolumn{5}{c|}{$\longleftarrow$ Total Number of Dataset $\longrightarrow$}\\
%	\cline{2-6}\revealcline
%	$1^{st}$ iteration & \ccell{1}{5}
%	\nextrow{2-6}
%	$2^{nd}$ iteration & \ccell{2}{5}
%	\nextrow{2-6}
%	$3^{rd}$ iteration & \ccell{3}{5}
%	\nextrow{2-6}
%	Experiment 4 & \ccell{4}{5}
%	\nextrow{2-6}
%	Experiment 5 & \ccell{5}{5}\\
%	\cline{2-6}
%\end{tabular}\hskip1em
%\begin{tabular}[c]{|p{2em}|l}
%	\cline{1-1}
%	\ccell{0}{1}  & Training
%	\nextrow{1-1}
%	\ccell{1}{1}  & Validation\\
%	\cline{1-1}
%\end{tabular}
%\end{frame}


\begin{frame}
\begin{figure}
	\includestandalone[width=8cm]{tikzFigs/singleChannelNetwork}%     without .tex extension
	% or use \input{mytikz}
	\caption{Single channel LSTM sequence classifier}
	\label{fig:single_channel}
\end{figure}
\end{frame}


\begin{frame}
\begin{figure}
	\includestandalone[width=8cm]{tikzFigs/featureFusionNetwork}%     without .tex extension
	% or use \input{mytikz}
	\caption{Single channel LSTM used for multivariate sequence classification}
	\label{fig:featureFusion}
\end{figure}
\end{frame}


\begin{frame}
\begin{figure}
	\includestandalone[width=8cm]{tikzFigs/weightedMultiChannel}%     without .tex extension
	% or use \input{mytikz}
	\caption{Weighted multichannel LSTM sequence classifier}
	\label{fig:weighted_channel}
\end{figure}
\end{frame}


\begin{frame}
%\frametitle{}
\begin{center}
	\noindent
	\renewcommand\arraystretch{1.5}
	\setlength\tabcolsep{0pt}
	
	\begin{tabular}{c >{\bfseries}r @{\hspace{0.7em}}c @{\hspace{0.4em}}c @{\hspace{0.7em}}l}
		\multirow{10}{*}{\rotatebox{90}{\parbox{1.1cm}{\bfseries\centering Actual\\  }}} & 
		& \multicolumn{2}{c}{\bfseries Predicted} & \\
		& & \bfseries Error & \bfseries No error & \bfseries  \\
		& Error & \MyBox{True}{Positive} & \MyBox{False}{Negative} & $ \sum P$\\[2.4em]
		& No error & \MyBox{False}{Positive} & \MyBox{True}{Negative} & $ \sum N$ \\
		&  & $\sum p$ & $ \sum n$ &
	\end{tabular}
\end{center}
\end{frame}


%\begin{frame}
%\frametitle{}
%\begin{center}
%\begin{tabular}{@{}cc cc@{}}
%	\multicolumn{1}{c}{} &\multicolumn{1}{c}{} &\multicolumn{2}{c}{Predicted} \\ 
%	\cmidrule(lr){3-4}
%	\multicolumn{1}{c}{} & 
%	\multicolumn{1}{c}{} & 
%	\multicolumn{1}{c}{Yes} & 
%	\multicolumn{1}{c}{No} \\ 
%	\cline{2-4}
%	\multirow[c]{2}{*}{\rotatebox[origin=tr]{90}{Actual}}
%	& Yes  & 100 & 0   \\[1.5ex]
%	& No  & 10   & 80 \\ 
%	\cline{2-4}
%\end{tabular}
%\end{center}
%\end{frame}

\end{document}
